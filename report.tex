% Standard 10pt
\documentclass[a4paper,twoside, openright]{report}

\usepackage[utf8]{inputenc}
\usepackage{fontspec, emptypage}
\usepackage[a4paper, left=2.5cm, right=2.5cm, bottom=2.8cm, top=2.8cm]{geometry}
\usepackage[ngerman,english]{babel}
\usepackage{fancyhdr, graphicx, amsmath, hyperref, longtable, float, enumitem, blindtext}
% For Bibliography
\usepackage{natbib}
\bibliographystyle{authordate1}
% For Abkürzungsverzeichnis
\usepackage{acronym}
% For appendix
\usepackage[toc,page]{appendix}
% For tikz
\usepackage{calc, ifthen, tikz}
% For including pdfs
\usepackage{pdfpages}
% Caption setup
\usepackage[tableposition=b]{caption}
\captionsetup[longtable]{skip=1em}
% Important for longtable
\usepackage{arydshln}

\hypersetup{
	colorlinks,
	citecolor=black,
	filecolor=black,
	linkcolor=black,
	urlcolor=black
}

\renewcommand{\baselinestretch}{1.2}
\newcommand{\tabitem}{~~\llap{\textbullet}~~}
\newcommand{\HRule}{\rule{\linewidth}{0.3mm}} % Defines a new command for the horizontal lines, change thickness here

\addto\captionsngerman{\let\appendixtocname\appendixname%
	\let\appendixpagename\appendixname}

\title{Bachelor Thesis}

\parindent 0pt

\begin{document}

% ----------------------------------------
% Titelblatt
% ----------------------------------------
\begin{titlepage}
\begingroup
\centering
\vspace*{\baselineskip}

\thispagestyle{empty}

\begin{figure}[H]
	\centering
	\begin{minipage}{.49\textwidth}
		\centering
		\includegraphics[scale=0.3]{img/InvestmentSolutions.jpg}
	\end{minipage}
\end{figure}

\vspace*{1\baselineskip}

\HRule \\[0.4cm]
{\LARGE \textbf{Entwicklung einer tabletbasierten Applikation zur Unterstützung der Arbeitspraktiken in der Hypothekarberatung}}
\HRule \\[1.5cm]

\vspace*{1\baselineskip}

{\large \textbf{Bachelor Thesis}}

\vspace*{1\baselineskip}

\scshape % Small caps
\large University of Zurich - Institut für Informatik \\ \vspace{3.5mm}
\large Information Management Research Group \\ \vspace{3.5mm}
\large Prof. Dr. Gerhard Schwabe

\vspace*{0.5\baselineskip}



\begin{align*}
\textbf{Verfasser:} &\ \text{Pascal Zehnder} \\
&\ \text{13-712-336} \\
&\ \text{pascal.zehnder@bf.uzh.ch} \\
\textbf{Geburtsort:} &\ \text{Schwyz} \\
\textbf{Studienrichtung:} &\ \text{Wirtschaftsinformatik} \\
\textbf{Betreuender Assistent:} &\ \text{Mehmet Kilic} \\
\textbf{Zweitbetreuer:} &\ \text{Mateusz Dolata} \\
\textbf{Abgabe der Arbeit:} &\ \text{19.04.2017} \\
\end{align*}

\endgroup
\end{titlepage}

\newpage

\pagenumbering{roman}

% ----------------------------------------
% Abstract auf English
% ----------------------------------------
\selectlanguage{english}%
\chapter*{Abstract}
In numerous industries modern technologies more and more find its application. Multiple surveys about advisory servies in the financial sector came to the conclusion that various banking institutions are still using pen and paper during an advisory service. IT-supported artefacts furthermore can't find their application in financial advisory services.\\

In previous works a fundamental acceptance of using IT-supported artefacts in financial advisory services could be recognized on the part of customer and advisor. In comparison to a conventional advisory service an additional benefit could be recognized.\\

Based on the analysis of a previous prototype and particular conventional advice giving, as well as interviews with customers and advisors, requirements for an optimally designed prototype were generated. The main focus of the task was on the development of the prototype. The implemented prototype was then evaluated in a focus group. The feedback of the participants confirmed the improvement of the usability and labor practices of the advisors which was a consequence of conscious adjustments and newly developed features.

% ----------------------------------------
% Inhaltsübersicht
% ----------------------------------------
\tableofcontents

\newpage

% ----------------------------------------
% Start des eigentlichen Inhalts
% ----------------------------------------
\pagestyle{headings}
\pagenumbering{arabic}
\section{Introduction}
\label{sec:introduction}

During their studies, students in the field of Banking and Finance learn a lot about asset management practices and theoretical business management aspects. To provide these students with an opportunity to apply their knowledge and understanding of the portfolio management process in practice, the Department of Banking of Finance offers the seminar ''Advanced Portfolio Management Game (S)'' every fall semester. During the seminar, students play a simulated game (''Portfolio management SIM'') in which they cooperate in teams to steer a bank's portfolio management and investment strategy as well as its business management of investment funds. Competition among the different student teams should make the learning process entertaining. \\

According to the Department of Banking and Finance, the game has the following key learning goals:
\begin{itemize}
  \item Students practice how money can be invested in financial markets in a systematic fashion.
  \item Students learn what factors make financial market forecasts possible and what limitations the applied models have for forecasting.
  \item Students learn which factors are relevant for the success (performance) of investments and can distinguish between factors that promise short-term and long-term success.
\end{itemize}

The game focuses on a structured investment process (as visualized in \Cref{fig:investment_process}). The process covers the steps from getting to know different customer types of the bank, selecting a suitable long-term Strategic Asset Allocation (SAA), adjusting it in the Tactical Asset Allocation (TAA) according to the status of the economy, and finally selecting appropriate titles matching SAA and TAA in the depot realization. In addition to these steps, the teams also decide on key business decisions like the salary to pay their employees and how much dividends to pay their shareholders.

\begin{figure}[h!]
  \centering
  \includegraphics[scale=0.6]{img/private_banking_process.png}
  \caption{Overview of the structured investment process}
  \label{fig:investment_process}
\end{figure}

The currently used version of the ''Portfolio management SIM'' is both technically and didactically outdated. It was initiated in 2005 and initially developed by the Department of Banking and Finance at the University of Zurich in cooperation with the Swiss bank Julius Baer as well as the simulation development company game solution ag. Rapid technological development since that time enables new perspectives and possibilities in the field of game-based learning, which lead to the ideation of for this master project. \\

% TODO insert old game screenshots? not sure if helpful

The herein described master project (further referred to as the ''pfm-game'' project) aims to redesign and reprogram the existing ''Portfolio management SIM'' such that it is state-of-the-art from both a technical and didactical point of view. The project team assembled from students of the Departement of Informations and supervisors from both the Departement of Banking and Finance and Department of Informatics decided to cooperate on a modern version of the Portfolio Management Game. The previous experience of the students in developing web applications, as well as their interest in analyzing financial processes and developing this game provides a solid foundation for the project. By re-developing the application, the Department of Banking and Finance wants to create a simulation of a typical portfolio management process that helps the students in their learning process by strengthening their practical decision making based on their theoretical knowledge.
\newpage

% ----------------------------------------
% Literaturverzeichnis
% ----------------------------------------
\selectlanguage{ngerman}
\bibliography{bachelor_thesis}


\end{document}
