\section{Methodology}
Different tools helped us to understand a typical investment process to model the game best for a proper learning of the students. Those tools are described in this section.

\subsection{Requirements Engineering}
The typical requirements engineering builds the base of our engineering design process. By creating user stories we had a common basis to define the requirements toegther with our principals from the DBF. As usual when definining user stories we classified those stories into following three categories: Nice-to-have, Shoul-have, Must-have. Additionally they are structured into different functional or organizational parts. All the stories can be found in the appendix~\ref{sec:appendix_user_stories}.
% TODO more

\subsection{User Interviews}
By interviewing multiple individuals we got an insight into different practices during the investment process in different companies, such as Zürcher Kantonalbank and Credit Suisse. They have showed some screens of their internal applications which helped us to design the depot relization part of the students decisions.

\subsection{Observation of Game Execution}
The game observation was separated in following parts:
\begin{itemize}
  \item Executive Education Students Observation during their final seminar at the Uetliberg
  \item Observation of different knowledge types in one room (Undergraduate studies, Graduate studies)
  \item Master Seminar: Advanced Portfolio Management Seminar
\end{itemize}

Multiple tools have been used by observing students playing the old game. % TODO bla (continue)

\subsection{Design and Iterative Prototyping}
Initially we designed some screens, using a sketching software, for the students process which includes the SAA, TAA, depot relization and business administration. As we realized that the game does not only consist of those screens we decided, due to time constraints, to start implementing screens by iterative prototyping.
% TODO more
